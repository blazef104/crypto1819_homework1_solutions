\documentclass[11pt]{article}
\usepackage[utf8]{inputenc}
\usepackage{color}
\usepackage{framed}
\usepackage{amsmath}
\usepackage{titletoc}
\usepackage{hyperref}
\usepackage{amsmath}
\usepackage{amsfonts}
\usepackage{amsthm}
\usepackage{amssymb}
\usepackage{thmtools}
\usepackage{acronym}
\usepackage{multirow}
\usepackage{cleveref}
\usepackage{graphicx}
\usepackage{mathtools}
\usepackage{cancel}
\usepackage{ifthen}
\usepackage{etoolbox}
\usepackage{xcolor}
\usepackage{subcaption}
\usepackage{calc}
\usepackage{tikz}
\hypersetup{colorlinks=false}

\makeatletter


\titlecontents{chapter}[0em]{\lsstyle\smallskip\bfseries}%\vspace{1cm}%
      {\contentslabel[\chaptername~\thecontentslabel]{0em}\hspace{5.6em}}%
      {\renewcommand\thecontentslabel{\relax}\itshape}%numberless%
      {\hfill\contentspage}[\medskip]%


%this is used to create space between \forall, \exist [ and others ] with the
      %next character
\let\eps\varepsilon
\newcommand\+{\mkern2mu}




\newcommand{\ie}{\textit{i.e.}, }
\newcommand{\eg}{\textit{e.g.}, }

\newcommand{\divides}{\big|}

\newcommand{\biglor}{\bigvee}
\newcommand{\bigxor}{\bigoplus}

\newcommand{\Definition}{\ensuremath{\mathop{=}\limits^{\scriptscriptstyle \triangle}}}

\renewcommand{\vec}[1]{\overline{#1}}
\newcommand{\norm}[1]{\left\lVert{#1}\right\rVert}

\renewcommand{\epsilon}{\varepsilon}

\newcommand{\comma}{\ensuremath{, \allowbreak}}
\newcommand{\from}{\leftarrow}
\newcommand{\toandfrom}{\rightleftarrows}

\newcommand{\A}{\mathcal{A}}
\newcommand{\B}{\mathcal{B}}
\newcommand{\C}{\mathcal{C}}
\newcommand{\D}{\mathcal{D}}
\newcommand{\F}{\mathcal{F}}
\newcommand{\G}{\mathcal{G}}
\renewcommand{\H}{\mathcal{H}}
\newcommand{\K}{\mathcal{K}}
\newcommand{\M}{\mathcal{M}}
\renewcommand{\P}{\mathcal{P}}
\newcommand{\R}{\mathcal{R}}
\newcommand{\U}{\mathcal{U}}
\newcommand{\V}{\mathcal{V}}
\newcommand{\X}{\mathcal{X}}
\newcommand{\Y}{\mathcal{Y}}
\newcommand{\Z}{\mathcal{Z}}

\newcommand{\RO}{\mathrm{RO}}

\newcommand{\Naturals}{\mathbb{N}^{+}}
\newcommand{\NaturalsZ}{\mathbb{N}}
\newcommand{\Integers}{\mathbb{Z}}
\newcommand{\Reals}{\mathbb{R}}
\newcommand{\Bool}{\{0,1\}}
\newcommand{\QR}[1][p]{\mathbb{QR}_{#1}}

\newcommand{\IntegersPrimeGroup}[1][p]{\Integers^{\star}_{#1}}

\newcommand{\Coll}{\mathrm{Coll}}
\newcommand{\Ext}{\mathrm{Ext}}
\newcommand{\Gen}{\mathrm{Gen}}
\newcommand{\KGen}{\mathrm{KGen}}
\newcommand{\KeyGen}{\mathrm{KeyGen}}
\newcommand{\Mac}{\mathrm{Mac}}
\newcommand{\Sign}{\mathrm{Sign}}
\newcommand{\Vrfy}{\mathrm{Vrfy}}
\newcommand{\Enc}{\mathrm{Enc}}
\newcommand{\Dec}{\mathrm{Dec}}
\newcommand{\Trans}{\mathrm{Trans}}

\renewcommand{\Game}{\G}

\newcommand{\IDSGame}{\Game^{\mathrm{id}}_{\Pi, \Adv}}

\newcommand{\GenericEncScheme}{\Pi}
\newcommand{\GenericKeyGen}{\Gen}
\newcommand{\GenericEnc}{\Enc}
\newcommand{\GenericDec}{\Dec}
\newcommand{\GenericEncSchemeTuple}{\GenericEncScheme = \left( \GenericKeyGen\comma \GenericEnc\comma \GenericDec \right)}

\newcommand{\GenericMACScheme}[1][]{\Pi_{#1}}
\newcommand{\GenericMac}{\Mac}
\newcommand{\GenericVrfy}{\Vrfy}
\newcommand{\GenericAuthSpace}{\Phi}
\newcommand{\GenericMACSchemeTuple}[1][]{\GenericMACScheme[#1] = \left( \GenericKeyGen\comma \GenericMac\comma \GenericVrfy \right)}

\newcommand{\SKEScheme}[1][]{\Pi_{#1}}
\newcommand{\SKEKeyGen}{\Gen}
\newcommand{\SKEEnc}{\Enc}
\newcommand{\SKEDec}{\Dec}
\newcommand{\SKESchemeTuple}[1][]{\SKEScheme[#1] = \left( \SKEKeyGen \comma \SKEEnc \comma \SKEDec \right)}

\newcommand{\SKEGameOneTime}{\Game^{\scriptstyle \text{one time}}_{\SKEScheme, \Adv}}

\newcommand{\CCAScheme}{\Pi'}
\newcommand{\CCAKeyGen}{\Gen'}
\newcommand{\CCAEnc}{\Enc'}
\newcommand{\CCADec}{\Dec'}
\newcommand{\CCASchemeTuple}{\CCAScheme = \left( \CCAKeyGen \comma \CCAEnc \comma \CCADec \right)}

\newcommand{\PRFGameReal}{\Game^{\scriptstyle \text{real}}_{\F, \Adv}}
\newcommand{\PRFGameRand}{\Game^{\scriptstyle \text{rand}}_{\F, \Adv}}

\newcommand{\PKEScheme}{\Pi}
\newcommand{\PKEKeyGen}{\Gen}
\newcommand{\PKEEnc}{\Enc}
\newcommand{\PKEDec}{\Dec}
\newcommand{\PKESchemeTuple}{\PKEScheme = \left( \PKEKeyGen \comma \PKEEnc \comma \PKEDec \right)}

\newcommand{\PKEGameCPA}{\Game_{\Adv, \Pi}^{\text{CPA \\ one-time}}}
\newcommand{\PKEGameCCA}[1]{\Game_{\Adv, \Pi}^{\text{CCA{#1}}}}

\newcommand{\SignScheme}{\Pi}
\newcommand{\SignSchemeKGen}{\KGen}
\newcommand{\SignSchemeSign}{\Sign}
\newcommand{\SignSchemeVrfy}{\Vrfy}
\newcommand{\SignSchemeTuple}{\SignScheme = \left( \SignSchemeKGen \comma \SignSchemeSign \comma \SignSchemeVrfy \right)}

\newcommand{\RSA}{\mathrm{RSA}}
\newcommand{\RSAGen}{\Gen_{\RSA}}
\newcommand{\RSAfun}{f_{\RSA}}
\newcommand{\RSAinv}{f_{\RSA}^{-1}}
\newcommand{\RSAtuple}{(\RSAGen, \RSAfun, \RSAinv)}

\newcommand{\CPAGame}{\Game^{\scriptstyle \text{cpa}}_{\SKEScheme,\Adv}}
\newcommand{\CCAGame}{\Game^{\scriptstyle \text{cca}}_{\SKEScheme,\Adv}}
\newcommand{\EAVGame}{\Game^{\scriptstyle \text{eav}}_{\SKEScheme,\Adv}}

\newcommand{\CFPTuple}{\Pi = (\Gen, f_0, f_1)}
\newcommand{\CFPGame}{\Game^{\scriptstyle \text{CF}}_{\Pi, \Adv}}

\newcommand{\CRHGame}{\Game^{\scriptstyle \text{CR}}_{\H,\Adv}}
\newcommand{\CRDMGame}{\Game^{\scriptstyle \text{CR}}_{{\scriptstyle \text{DM}}, \Adv}}

\newcommand{\UFCMAGame}{\Game^{\scriptstyle \text{ufcma}}_{\GenericMACScheme,\Adv}}

\newcommand{\UHFGameReal}{\Game^{\scriptstyle \text{real}}_{\F(\H),\Adv}}
\newcommand{\UHFGameHybrid}{H_{\$,\H,\Adv}}
\newcommand{\UHFGameRand}{\Game^{\scriptstyle \text{rand}}_{\$,\Adv}}

\newcommand{\OTPEnc}{\Enc}
\newcommand{\OTPDec}{\Dec}

\newcommand{\Feistel}[1][F]{\psi_{#1}}

\newcommand{\Group}{\mathbb{G}}
\newcommand{\Field}{\mathbb{F}}
\newcommand{\GroupTuple}{(\Group \comma g \comma q)}
\newcommand{\GroupGen}{\GroupTuple \from \mathrm{GroupGen}(1^{\lambda})}

\newcommand{\Bilin}{\mathrm{Bilin}}
\newcommand{\BilinearGroup}{\Group}
\newcommand{\BilinearGroupBis}{\BilinearGroup_{T}}
\newcommand{\BilinearMap}{\hat{e}}
\newcommand{\BilinearGroupTuple}{\left( \BilinearGroup, \BilinearGroupBis, q, g, \BilinearMap \right)}

\newcommand{\WatersKGen}{\mathrm{BiGen}}

\newcommand{\pk}{pk}
\newcommand{\sk}{sk}

\newcommand{\params}{params}

\newcommand{\pr}{\mathop{\mathrm{Pr}}}
\renewcommand{\Pr}[2][]{\ensuremath{\pr\limits_{#1} \left[ {#2} \right]}}

\newcommand{\rand}[1]{\from {\scriptstyle \$} {#1}}
\newcommand{\abs}[1]{\left|{#1}\right|}
\newcommand{\poly}{\mathrm{poly}}

\newcommand{\DotProduct}[2]{\left< {#1}, {#2} \right>}
\newcommand{\Generator}[1]{\left< {#1} \right>}

\newcommand{\repr}[1]{\left< {#1} \right>}

\newcommand{\CompInd}{\approx_c}
\newcommand{\StatInd}{\approx_s}

\newcommand{\Adv}{\A}
\newcommand{\Distinguisher}{\D}

\newcommand{\negl}[1]{\mathrm{negl}\left({#1}\right)}

\newcommand{\xor}{\oplus}
%========================================================================================================



%\renewcommand\thmt@mklistcmd{%
%  \@xa\protected@edef\csname l@\thmt@envname\endcsname{% CHECK: why p@edef?
%    \@nx\@dottedtocline{1}{1.5em}{\@nx\thmt@listnumwidth}{\thmt@thmname}{mu}%
%  }%
%  \ifthmt@isstarred
%    \@xa\def\csname ll@\thmt@envname\endcsname{%
%      \protect\numberline{\thmt@thmname\protect\let\protect\autodot\protect:}%
%      \ifx\@empty\thmt@shortoptarg\else\protect\thmtformatoptarg{\thmt@shortoptarg}\fi
%    }%
%  \else
%    \@xa\def\csname ll@\thmt@envname\endcsname{%
%      % \thmt@thmname\ \csname the\thmt@envname\endcsname: \hfil%
%      \ifx\@empty\thmt@shortoptarg\else\thmt@shortoptarg\fi
%    }%
%  \fi
%  \@xa\gdef\csname thmt@contentsline@\thmt@envname\endcsname{%
%    \thmt@contentslineShow% default:show
%  }%
%}
\makeatother


\theoremstyle{plain}
\declaretheorem[qed=\ensuremath{\diamond}]{theorem}
\declaretheorem[qed=\ensuremath{\diamond}]{lemma}
\declaretheorem[qed=\ensuremath{\diamond}]{corollary}
\declaretheorem[qed=\ensuremath{\diamond}]{fact}
\declaretheorem[qed=\ensuremath{\diamond}]{claim}
\declaretheorem[qed=\ensuremath{\diamond}]{observation}
\declaretheorem[qed=\ensuremath{\diamond}]{proposition}

\crefname{thm}{theorem}{theorems}
\Crefname{thm}{Theorem}{Theorems}

\crefname{lem}{lemma}{lemmas}
\Crefname{lem}{Lemma}{Lemmas}

\crefname{cor}{corollary}{corollarys}
\Crefname{cor}{Corollary}{Corollarys}

\crefname{fct}{fact}{facts}
\Crefname{fct}{Fact}{Facts}

\crefname{clm}{claim}{claims}
\Crefname{clm}{Claim}{Claims}

\crefname{obs}{observation}{observations}
\Crefname{obs}{Observation}{Observations}

\crefname{prop}{proposition}{propositions}
\Crefname{prop}{Proposition}{Propositions}


\theoremstyle{definition}
\declaretheorem[qed=\ensuremath{\diamond}]{definition}
\declaretheorem[qed=\ensuremath{\diamond}]{construction}

\crefname{defn}{definition}{definitions}
\Crefname{defn}{Definition}{Definitions}

\crefname{cons}{construction}{constructions}
\Crefname{cons}{Construction}{Constructions}


% \theoremstyle{plain}
\newtheorem{thm}{Theorem}
\newtheorem{lem}{Lemma}
\newtheorem{cor}{Corollary}
\newtheorem{fct}{Fact}
\newtheorem{clm}{Claim}
\newtheorem{obs}{Observation}
\newtheorem{prop}{Proposition}
\newtheorem{question}{Question}
\newtheorem{exercise}[theorem]{Exercise}
\newtheorem{solution}{Solution}
\newtheorem{example}{Example}
% \theoremstyle{definition}
\newtheorem{defn}{Definition}
\newtheorem{cons}{Construction}

%----------------------IMAGES-----------------------------------
\usepackage[underline=false]{pgf-umlsd}
\usepackage{tikz}

% From http://tex.stackexchange.com/questions/164707/how-to-use-greek-letters-in-pgf-umlsd-or-generally-terms-starting-with
\renewcommand{\mess}[4][0]{
  \stepcounter{seqlevel}
  \path
  (#2)+(0,-\theseqlevel*\unitfactor-0.7*\unitfactor) node (mess from) {};
  \addtocounter{seqlevel}{#1}
  \path
  (#4)+(0,-\theseqlevel*\unitfactor-0.7*\unitfactor) node (mess to) {};
  \draw[->,>=angle 60] (mess from) -- (mess to) node[midway, above]
  {#3};
  \node (\detokenize{#3} from) at (mess from) {};
  \node (\detokenize{#3} to) at (mess to) {};
}
% From http://tex.stackexchange.com/questions/98525/pgf-umlsd-and-externalize
\newcommand{\sdinit}{%
   \pgfdeclarelayer{umlsd@background}%
   \pgfdeclarelayer{umlsd@threadlayer}%
   \pgfsetlayers{umlsd@background,umlsd@threadlayer,main}%
}
\newcommand{\sdbegin}{%
   \setlength{\unitlength}{1cm}%
   \tikzstyle{sequence}=[coordinate]%
   \tikzstyle{inststyle}=[rectangle, draw, anchor=west, minimum
   height=0.8cm, minimum width=1.6cm, fill=white, 
   drop shadow={opacity=1,fill=black}]%
   \ifpgfumlsdroundedcorners%
      \tikzstyle{inststyle}+=[rounded corners=3mm]%
   \fi%
   \tikzstyle{blockstyle}=[anchor=north west]%
   \tikzstyle{blockcommentstyle}=[anchor=north west, font=\small]%
   \tikzstyle{dot}=[inner sep=0pt,fill=black,circle,minimum size=0.2pt]%
   \global\def\unitfactor{0.6}%
   \global\def\threadbias{center}%
   % reset counters
   \setcounter{preinst}{0}%
   \setcounter{instnum}{0}%
   \setcounter{threadnum}{0}%
   \setcounter{seqlevel}{0}%
   \setcounter{callevel}{0}%
   \setcounter{callselflevel}{0}%
   \setcounter{blocklevel}{0}%
   % origin
   \node[coordinate] (inst0) {};%
}
\newcommand{\sdend}{%
   \begin{pgfonlayer}{umlsd@background}%
      \ifnum\value{instnum}>0%
         \foreach \t [evaluate=\t] in {1,...,\theinstnum}{%
            \draw[dotted] (inst\t) -- ++(0,-\theseqlevel*\unitfactor-2.2*\unitfactor);%
         }%
      \fi%
      \ifnum\value{threadnum}>0%
         \foreach \t [evaluate=\t] in {1,...,\thethreadnum}{%
            \path (thread\t)+(0,-\theseqlevel*\unitfactor-0.1*\unitfactor) node (threadend) {};%
            \tikzstyle{threadstyle}+=[threadcolor\t]%
            \drawthread{thread\t}{threadend}%
         }%
      \fi%
   \end{pgfonlayer}%
}



\newcounter{listcount} \newcounter{totcount}
\newcommand{\printarray}[2][1em]{% \printarray[<width>]{<array list>}
  \unskip \setcounter{totcount}{0}% Reset totcount counter
  \renewcommand*{\do}[1]{\stepcounter{totcount}}% Count elements
  \docsvlist{#2}% Process list a first time to obtain # of elements
  \setcounter{listcount}{0}% Reset listcount counter
  \renewcommand*{\do}[1]{%
    \stepcounter{listcount}% Move to next element
    \framebox[#1][c]{\rule{0pt}{1.5ex}\smash{\ensuremath{##1}}}%
    \ifnum\value{listcount}<\value{totcount}\thickspace\fi
  }
  \docsvlist{#2}% Process list a second time to typeset each element
}


% \newcommand{\thmsymbol}{\( \diamond \)}
% \newenvironment{definition}{\begin{defn}%
% \renewcommand{\qedsymbol}{\thmsymbol}\pushQED{\qed}}%
% {\popQED\end{defn}}
% \newenvironment{construction}{\begin{cons}%
% \renewcommand{\qedsymbol}{\thmsymbol}\pushQED{\qed}}%
% {\popQED\end{cons}}
% \newenvironment{theorem}{\begin{thm}%
% \renewcommand{\qedsymbol}{\thmsymbol}\pushQED{\qed}}%
% {\popQED\end{thm}}
% \newenvironment{lemma}{\begin{lem}%
% \renewcommand{\qedsymbol}{\thmsymbol}\pushQED{\qed}}%
% {\popQED\end{lem}}
% \newenvironment{corollary}{\begin{cor}%
% \renewcommand{\qedsymbol}{\thmsymbol}\pushQED{\qed}}%
% {\popQED\end{cor}}
% \newenvironment{fact}{\begin{fct}%
% \renewcommand{\qedsymbol}{\thmsymbol}\pushQED{\qed}}%
% {\popQED\end{fct}}
% \newenvironment{claim}{\begin{clm}%
% \renewcommand{\qedsymbol}{\thmsymbol}\pushQED{\qed}}%
% {\popQED\end{clm}}
% \newenvironment{observation}{\begin{obs}%
% \renewcommand{\qedsymbol}{\thmsymbol}\pushQED{\qed}}%
% {\popQED\end{obs}}
% \newenvironment{proposition}{\begin{prop}%
% \renewcommand{\qedsymbol}{\thmsymbol}\pushQED{\qed}}%
% {\popQED\end{prop}}


% \renewcommand{\thmtformatoptarg}[1]{ #1}
\setcounter{chapter}{4}


\newcounter{t0d0_counter}
\newcommand{\notodo}[1]{
}
\newcommand{\todo}[1]{
  \stepcounter{t0d0_counter}
  \definecolor{shadecolor}{rgb}{1,1,0} % this is yellow
  \begin{shaded}
  T0D0 \arabic{t0d0_counter}: #1
  \end{shaded}
}


\title{\textbf{Cryptography Homework 1}\\ {\normalsize (Solutions)}}
\author{Giulio Ginesi}
\date{}


\begin{document}

\maketitle

\section{Perfect Secrecy}

\todo{Image of the game, review the demonstration we have done}

We want to prove that (a) $Pr[Game_{\pi,A}^{eav}=1]=\frac{1}2$ is perfectly secret. 
Let's use the last one of the 3 properties we saw in class: 

(b) $Pr[Enc(K,m_0)=c]=Pr[Enc(K,m_1)=c] $.

We want to prove that $a \implies b$ and also $ b \implies a $
\bigskip
$a \implies b$

$\forall m_0,m_1 \in M, \forall c \in C $ we can rewrite the second equation using Game: 

$$ Pr[Game=1 | b=0]Pr[b=0]=Pr[Game=1 | b=1]Pr[b=1] $$

Since Game and b are independent we can use Bayes as follows:

$$\Rightarrow \frac{Pr[Game_{\pi, A}^{eav}=1]Pr[b=0]}{Pr[b=0]}Pr[b=0]=\frac{Pr[Game_{\pi, A}^{eav}=1]Pr[b=1]}{Pr[b=1]}Pr[b=1]$$
$$\Rightarrow Pr[Game_{\pi, A}^{eav}=1]Pr[b=0]=Pr[Game_{\pi, A}^{eav}=1]Pr[b=1]$$



\section{Universal Hashing}
 \todo{Maybe we should try and go through all what we have done so far}
 (a)

 (i)  
 We can prove that $t-wise \implies (t-1)-wise$ just by writing down the notion of $t$-wise independence.

 $$ \sum_{1=0}^t{Pr[h_s(x_i)=y_i]} = \sum_{i=0}^{t-1}{(Pr[h_s(x_i)=y_i])}+Pr[h_s(x_t)=y_t]$$

 From the above equation we can see that the $t-1$-wise independence is contained in the $t$ wise one. 

 (ii) $$ Pr[h_s(x_0)=y \wedge h_s(x_1)=y \wedge h_s(x_2)=y] = \frac{1}{|y^3|} $$
      
Now for 3 different x: $x_0 \neq x_1 \neq x_2$ the corresponding hash functions are as follows:

$$ h_s(x_0) = S_0 + S_1x_0 + S_2x_0^2 $$
$$ h_s(x_1) = S_0 + S_1x_1 + S_2x_1^2 $$
$$ h_s(x_2) = S_0 + S_1x_2 + S_2x_2^2 $$

$$ Pr_{(S_0,S_1,S_2)\gets Z_q^3}[h(x_0)= \varphi \wedge h(x_1)=\varphi \prime \wedge h(x_2)=\varphi \prime \prime]=$$

$$=Pr_{(S_0,S_1,S_2)\gets Z_q^3}[S_0+S_1x_0+S_2x_0^2=\varphi \wedge S_0+S_1x_1+S_2x_1^2=\varphi \prime \wedge S_0+S_1x_2+S_2x_2^2=\varphi \prime \prime]=$$


\[
=Pr_{(S_0,S_1,S_2)\gets Z_q^3}[
\begin{pmatrix}
	1 & x_0 & x_0^2 \\
	1 & x_1 & x_1^2 \\
	1 & x_2 & x_2^2
\end{pmatrix}
\cdot
\begin{pmatrix}
	S_0 \\
	S_1\\
	S_2
\end{pmatrix}
=
\begin{pmatrix}
\varphi \\
\varphi \prime \\
\varphi \prime \prime
\end{pmatrix}
]=
\]

\[
=Pr_{(S_0,S_1,S_2)\gets Z_q^3}[
\begin{pmatrix}
	S_0 \\
	S_1\\
	S_2
\end{pmatrix}
=
\begin{pmatrix}
	1 & x_0 & x_0^2 \\
	1 & x_1 & x_1^2 \\
	1 & x_2 & x_2^2
\end{pmatrix}^{-1}
\cdot
\begin{pmatrix}
\varphi \\
\varphi \prime \\
\varphi \prime \prime
\end{pmatrix}
]=
\]

$$= \frac{1}{|Z_q^3|}=\frac{1}{q^3}$$

(b)

(i) $ l = 128 $ what is the minimal min-entropy to achieve $\varepsilon = 2^{-80}$?

$$128 \leq k-(2log2^{80} -2)$$
$$k \geq 286$$

Entropy loss: $\delta = 158$

(ii)

If $ k = 238 $ what is the maximum amount of uniform randomness with $\varepsilon = 2^{-80}$?

$$l \leq 238-(2log2^{80}-2)$$
$$l \leq 80$$

Making computational assumptions explain how to obtain $ l = 320 $

Let's assume a larg enough $\varepsilon = 2^{-128}$ then our $k$ will be:
$$320 \leq k - (2log2^{128}-2)$$
$$k \geq 574$$ 

\todo{Review computational assumptions}

\section{One-Way Functions}
(a)
\todo{}
(b) \\
(i) By contraddiction assume that $\exists$ an adversary $A_g$ that is able to invert $g$ with non negligible probability.
We will then have:
$$ Pr[g(x \prime || j)=y : x \leftarrow\$ \{0,1\}^n, y\prime = g(x), |j|=log n, x\prime \leftarrow\$A_g(1^{\lambda)},y)] \geq \frac{1}{p(\lambda)}$$
We can also define another adversary $A_f$ which is able to invert $f$ with non negligible probability such that:
$$ A_f(y)=(A_g(y,j,x_j))$$
Note that $A_g$ reveals $g$ when taking $(y,j,x_j)$ as input and in the same way, $A_f$ reveals $f$ on input $y$.
Thereofre $A_f$ leaks f (on input $y$) with the same probability of $A_g$ which is non negligible. $\implies$ Contraddiction.

\todo{Review the part (ii) it is a bit over complicated}
(ii) Let $g$ be a OWF then define $f$ as follows:
$$ f(U_{n+logn})=f(U_n,U_{log n})=g(U_n),U_{log n},[U_n]_{U_{log n}}$$
where $[x\prime]_i$ denote the $i^{th}$ bit of $x\prime$.
In words, f receives an input on length $n+log n$ and outputs $g$ applied to the first $n$ bits, together with $i$ and the $i^{th}$ bit if the input ( $i$ is the value in the last $log n$ bits of the input).\\
Fix $i$, we construct an $A_i$ s.t. $A_i(f(x\prime))$ outputs $x\prime _i$ with $Pr=\frac{1}{2}+\frac{1}{2n}$.
$A_i$ receives $f(U_{n+log n}$ and checks the second $log n$ bits of the output:\\
If they encode the value $i$ then $A_i$ outputs these last bits.\\
Otherwise $A_i$ outputs a random bit $b \in {0,1}$. \\
Now 

$$Pr[A_i(f(U_n,U_{logn}))=[U_n]_i]=$$
$$=\frac{1}{2}Pr[U_{log n} \neq i]+1Pr[U_{log n}=i]=$$
$$=\frac{1}{2}(1-\frac{1}{n})=\frac{1}{2}+\frac{1}{2n} > \frac{1}{2} + \frac{1}{2(n+log n)}$$

We conclude that with inputs of lenth $n$, $A_i$ succeeds in the guess of $A^{th}$ with $Pr$ at least $\frac{1}{2}+\frac{1}{2n}$.

Now we just need to prove that $f$ is one way:\\
Assume by contraddiction that $\exists$ a PPT adversary $A$ which can invert $f$ with non negligible probability.
\todo{I think we can draw a figure of the game here (and maybe also above?)}

\section{Pseudorandom Generators}
\todo{}
\section{Pseudorandom Functions}
\todo{}
\section{Secret Key Encryption}
\todo{}
\section{Message Authentication}
\todo{}
(b)

(i)
CBC-MAC for VLM is not secure. Consider the following construction:

- Obtain a message $m_0\leftarrow\$\{0,1\}^n$ of 1 block and his Tag $\varphi_0 =F_k(m_0)$
I can construct a message $m^\star=(m_0,m_0\xor \phi_0)$ and a valid Tag $\varphi^\star=\varphi_0$.
In this way when the challenger will verify the message he will compute $\varphi^\star\prime=F_k(m_0\xor\varphi_0\xor\varphi_0)=F_k(m_0)=\varphi_0$. 
And then clearly $\varphi^\star=\varphi^\star\prime$

(ii)
CBC-MAC using randomness as initialization vector. This isn't secure, consider the following construction:

- Obtain 2 messages $m_0, m_1$ with $m_0\neq m_1$ and their tags $\varphi_0=(r_0,F_k(m_0\xor r_0)), \varphi_1=(r_1,F_k(m_1\xor r_1))$.
I can now forge a new message $m^\star=(m_0\xor m_1)$ and a valid Tag $\varphi^\star=(r_1\xor m_0, \varphi_1)$.
When the challenger will verify $\varphi^\star$ he will compute
$\varphi^\star\prime=F_k(m_1\xor m_0\xor r_1\xor m_0)=F_k(m_1\xor r_1)=\varphi_1$
therefore $\varphi^\star\prime=\varphi^\star$.

(iii) 
CBC-MAC that outputs the tag for each block $\varphi=\varphi_1,\varphi_2,...,\varphi_t$ is not secure. Consider the following construction:

- Obtain 2 messages $m_0=(m_{0,1},m_{0,2}), m_1=(m_{1,1},m_{1,2})$, of at least 2 blocks, and their tags 
$\varphi_0=(\varphi_{0,1},\varphi_{0,2}), \varphi_1=(\varphi_{1,1},\varphi_{1,2})$. Now i can construct 
$m^\star=(m_{1,1}, m_{2,1}\xor\varphi_{1,1}\xor\varphi_{2,1})$ and a valid Tag 
$\varphi^\star=(\varphi_{1,1}, \varphi_{2,2})$. When the challenger will verify $\varphi^\star$ he will compute
$\varphi^\star\prime=F_k(m_{1,1}),F_k(m_{2,1}\xor\varphi_{1,1}\xor\varphi_{2,1}\xor\varphi_{1,1})=\varphi_{2,2}$
therefore $\varphi^\star=\varphi^\star\prime$.
\end{document}
