\documentclass[11pt]{article}
\usepackage[utf8]{inputenc}
\usepackage{color}
\usepackage{framed}
\usepackage{amsmath}
\input{macro.tex}

\newcounter{t0d0_counter}
\newcommand{\notodo}[1]{
}
\newcommand{\todo}[1]{
  \stepcounter{t0d0_counter}
  \definecolor{shadecolor}{rgb}{1,1,0} % this is yellow
  \begin{shaded}
  T0D0 \arabic{t0d0_counter}: #1
  \end{shaded}
}


\title{\textbf{Cryptography Homework 1}\\ {\normalsize (Solutions)}}
\author{Giulio Ginesi}
\date{}


\begin{document}

\maketitle

\section{Perfect Secrecy}

\todo{Image of the game, review the demonstration we have done}

We want to prove that (a) $Pr[Game_{\pi,A}^{eav}=1]=\frac{1}2$ is perfectly secret. 
Let's use the last one of the 3 properties we saw in class: 

(b) $Pr[Enc(K,m_0)=c]=Pr[Enc(K,m_1)=c] $.

We want to prove that $a \implies b$ and also $ b \implies a $
\bigskip
$a \implies b$

$\forall m_0,m_1 \in M, \forall c \in C $ we can rewrite the second equation using Game: 

$$ Pr[Game=1 | b=0]Pr[b=0]=Pr[Game=1 | b=1]Pr[b=1] $$

Since Game and b are independent we can use Bayes as follows:

$$\Rightarrow \frac{Pr[Game_{\pi, A}^{eav}=1]Pr[b=0]}{Pr[b=0]}Pr[b=0]=\frac{Pr[Game_{\pi, A}^{eav}=1]Pr[b=1]}{Pr[b=1]}Pr[b=1]$$
$$\Rightarrow Pr[Game_{\pi, A}^{eav}=1]Pr[b=0]=Pr[Game_{\pi, A}^{eav}=1]Pr[b=1]$$



\section{Universal Hashing}
 \todo{Maybe we should try and go through all what we have done so far}
 (a)

 (i)  
 We can prove that $t-wise \implies (t-1)-wise$ just by writing down the notion of $t$-wise independence.

 $$ \sum_{1=0}^t{Pr[h_s(x_i)=y_i]} = \sum_{i=0}^{t-1}{(Pr[h_s(x_i)=y_i])}+Pr[h_s(x_t)=y_t]$$

 From the above equation we can see that the $t-1$-wise independence is contained in the $t$ wise one. 

 (ii) $$ Pr[h_s(x_0)=y \wedge h_s(x_1)=y \wedge h_s(x_2)=y] = \frac{1}{|y^3|} $$
      
Now for 3 different x: $x_0 \neq x_1 \neq x_2$ the corresponding hash functions are as follows:

$$ h_s(x_0) = S_0 + S_1x_0 + S_2x_0^2 $$
$$ h_s(x_1) = S_0 + S_1x_1 + S_2x_1^2 $$
$$ h_s(x_2) = S_0 + S_1x_2 + S_2x_2^2 $$

$$ Pr_{(S_0,S_1,S_2)\gets Z_q^3}[h(x_0)= \varphi \wedge h(x_1)=\varphi \prime \wedge h(x_2)=\varphi \prime \prime]=$$

$$=Pr_{(S_0,S_1,S_2)\gets Z_q^3}[S_0+S_1x_0+S_2x_0^2=\varphi \wedge S_0+S_1x_1+S_2x_1^2=\varphi \prime \wedge S_0+S_1x_2+S_2x_2^2=\varphi \prime \prime]=$$


\[
=Pr_{(S_0,S_1,S_2)\gets Z_q^3}[
\begin{pmatrix}
	1 & x_0 & x_0^2 \\
	1 & x_1 & x_1^2 \\
	1 & x_2 & x_2^2
\end{pmatrix}
\cdot
\begin{pmatrix}
	S_0 \\
	S_1\\
	S_2
\end{pmatrix}
=
\begin{pmatrix}
\varphi \\
\varphi \prime \\
\varphi \prime \prime
\end{pmatrix}
]=
\]

\[
=Pr_{(S_0,S_1,S_2)\gets Z_q^3}[
\begin{pmatrix}
	S_0 \\
	S_1\\
	S_2
\end{pmatrix}
=
\begin{pmatrix}
	1 & x_0 & x_0^2 \\
	1 & x_1 & x_1^2 \\
	1 & x_2 & x_2^2
\end{pmatrix}^{-1}
\cdot
\begin{pmatrix}
\varphi \\
\varphi \prime \\
\varphi \prime \prime
\end{pmatrix}
]=
\]

$$= \frac{1}{|Z_q^3|}=\frac{1}{q^3}$$

(b)

(i) $ l = 128 $ what is the minimal min-entropy to achieve $\varepsilon = 2^{-80}$?

$$128 \leq k-(2log2^{80} -2)$$
$$k \geq 286$$

Entropy loss: $\delta = 158$

(ii)

If $ k = 238 $ what is the maximum amount of uniform randomness with $\varepsilon = 2^{-80}$?

$$l \leq 238-(2log2^{80}-2)$$
$$l \leq 80$$

Making computational assumptions explain how to obtain $ l = 320 $

Let's assume a larg enough $\varepsilon = 2^{-128}$ then our $k$ will be:
$$320 \leq k - (2log2^{128}-2)$$
$$k \geq 574$$ 

\todo{Review computational assumptions}

\section{One-Way Functions}
(a)
\todo{}
(b) \\
(i) By contraddiction assume that $\exists$ an adversary $A_g$ that is able to invert $g$ with non negligible probability.
We will then have:
$$ Pr[g(x \prime || j)=y : x \leftarrow\$ \{0,1\}^n, y\prime = g(x), |j|=log n, x\prime \leftarrow\$A_g(1^{\lambda)},y)] \geq \frac{1}{p(\lambda)}$$
We can also define another adversary $A_f$ which is able to invert $f$ with non negligible probability such that:
$$ A_f(y)=(A_g(y,j,x_j))$$
Note that $A_g$ reveals $g$ when taking $(y,j,x_j)$ as input and in the same way, $A_f$ reveals $f$ on input $y$.
Thereofre $A_f$ leaks f (on input $y$) with the same probability of $A_g$ which is non negligible. $\implies$ Contraddiction.

\todo{Review the part (ii) it is a bit over complicated}
(ii) Let $g$ be a OWF then define $f$ as follows:
$$ f(U_{n+logn})=f(U_n,U_{log n})=g(U_n),U_{log n},[U_n]_{U_{log n}}$$
where $[x\prime]_i$ denote the $i^{th}$ bit of $x\prime$.
In words, f receives an input on length $n+log n$ and outputs $g$ applied to the first $n$ bits, together with $i$ and the $i^{th}$ bit if the input ( $i$ is the value in the last $log n$ bits of the input).\\
Fix $i$, we construct an $A_i$ s.t. $A_i(f(x\prime))$ outputs $x\prime _i$ with $Pr=\frac{1}{2}+\frac{1}{2n}$.
$A_i$ receives $f(U_{n+log n}$ and checks the second $log n$ bits of the output:\\
If they encode the value $i$ then $A_i$ outputs these last bits.\\
Otherwise $A_i$ outputs a random bit $b \in {0,1}$. \\
Now 

$$Pr[A_i(f(U_n,U_{logn}))=[U_n]_i]=$$
$$=\frac{1}{2}Pr[U_{log n} \neq i]+1Pr[U_{log n}=i]=$$
$$=\frac{1}{2}(1-\frac{1}{n})=\frac{1}{2}+\frac{1}{2n} > \frac{1}{2} + \frac{1}{2(n+log n)}$$

We conclude that with inputs of lenth $n$, $A_i$ succeeds in the guess of $A^{th}$ with $Pr$ at least $\frac{1}{2}+\frac{1}{2n}$.

Now we just need to prove that $f$ is one way:\\
Assume by contraddiction that $\exists$ a PPT adversary $A$ which can invert $f$ with non negligible probability.
\todo{I think we can draw a figure of the game here (and maybe also above?)}

\section{Pseudorandom Generators}
\todo{}
\section{Pseudorandom Functions}
\todo{}
\section{Secret Key Encryption}
\todo{}
\section{Message Authentication}
\todo{}
(b)

(i)
CBC-MAC for VLM is not secure. Consider the following construction:

- Obtain a message $m_0\leftarrow\$\{0,1\}^n$ of 1 block and his Tag $\varphi_0 =F_k(m_0)$
I can construct a message $m^\star=(m_0,m_0\xor \phi_0)$ and a valid Tag $\varphi^\star=\varphi_0$.
In this way when the challenger will verify the message he will compute $\varphi^\star\prime=F_k(m_0\xor\varphi_0\xor\varphi_0)=F_k(m_0)=\varphi_0$. 
And then clearly $\varphi^\star=\varphi^\star\prime$

(ii)
CBC-MAC using randomness as initialization vector. This isn't secure, consider the following construction:

- Obtain 2 messages $m_0, m_1$ with $m_0\neq m_1$ and their tags $\varphi_0=(r_0,F_k(m_0\xor r_0)), \varphi_1=(r_1,F_k(m_1\xor r_1))$.
I can now forge a new message $m^\star=(m_0\xor m_1)$ and a valid Tag $\varphi^\star=(r_1\xor m_0, \varphi_1)$.
When the challenger will verify $\varphi^\star$ he will compute
$\varphi^\star\prime=F_k(m_1\xor m_0\xor r_1\xor m_0)=F_k(m_1\xor r_1)=\varphi_1$
therefore $\varphi^\star\prime=\varphi^\star$.

(iii) 
CBC-MAC that outputs the tag for each block $\varphi=\varphi_1,\varphi_2,...,\varphi_t$ is not secure. Consider the following construction:

- Obtain 2 messages $m_0=(m_{0,1},m_{0,2}), m_1=(m_{1,1},m_{1,2})$, of at least 2 blocks, and their tags 
$\varphi_0=(\varphi_{0,1},\varphi_{0,2}), \varphi_1=(\varphi_{1,1},\varphi_{1,2})$. Now i can construct 
$m^\star=(m_{1,1}, m_{2,1}\xor\varphi_{1,1}\xor\varphi_{2,1})$ and a valid Tag 
$\varphi^\star=(\varphi_{1,1}, \varphi_{2,2})$. When the challenger will verify $\varphi^\star$ he will compute
$\varphi^\star\prime=F_k(m_{1,1}),F_k(m_{2,1}\xor\varphi_{1,1}\xor\varphi_{2,1}\xor\varphi_{1,1})=\varphi_{2,2}$
therefore $\varphi^\star=\varphi^\star\prime$.
\end{document}
